% ============================================
% HƯỚNG DẪN SỬ DỤNG BEAMER TEMPLATE
% ============================================
% 
% QUAN TRỌNG: Template này dùng XeLaTeX để hỗ trợ tiếng Việt tốt
%
% 1. THAY ĐỔI THÔNG TIN CƠ BẢN:
%    - Dòng 40: Tiêu đề bài trình bày
%    - Dòng 41: Tiêu đề phụ (nếu có)
%    - Dòng 42: Tên tác giả
%    - Dòng 43: Tổ chức/Trường học
%    - Dòng 44: Ngày trình bày
%
% 2. THAY ĐỔI THEME & MÀU SẮC:
%    - Dòng 30: Chọn theme (Madrid, Berkeley, Singapore, Copenhagen, etc.)
%    - Dòng 31: Chọn color theme (default, beaver, beetle, crane, dove, etc.)
%
% 3. FONT TIẾNG VIỆT (optional):
%    Nếu muốn dùng font cụ thể, thêm sau dòng 36:
%    \setmainfont{Times New Roman}
%    \setsansfont{Arial}
%
% 4. THÊM/XÓA SLIDE:
%    - Mỗi \begin{frame}...\end{frame} là một slide
%    - Copy/paste block frame để tạo slide mới
%
% 5. THÊM HÌNH ẢNH:
%    - Upload ảnh vào project (drag & drop)
%    - Sử dụng: \includegraphics[width=0.5\textwidth]{ten-anh.jpg}
%
% 6. THÊM CODE:
%    - Sử dụng môi trường verbatim hoặc lstlisting
%    - Cần thêm \usepackage{listings} nếu dùng lstlisting
%
% ============================================

\documentclass{beamer}

% Theme và color
\usetheme{Madrid}          % Các theme khác: Berkeley, Singapore, Copenhagen, Boadilla
\usecolortheme{default}    % Các color khác: beaver, beetle, crane, dove, wolverine

% Packages
\usepackage{fontspec}       % For XeLaTeX - Unicode font support
\usepackage{polyglossia}    % Modern replacement for babel
\setdefaultlanguage{vietnamese}
% Để XeLaTeX tự động chọn font mặc định hỗ trợ Unicode
\usepackage{graphicx}
\usepackage{amsmath}
\usepackage{hyperref}

% Thông tin bài trình bày
\title{Tiêu Đề Bài Trình Bày}
\subtitle{Tiêu đề phụ (nếu có)}
\author{Tên Tác Giả}
\institute{Tên Trường/Tổ Chức}
\date{\today}

% Logo (optional - uncomment và thay đổi nếu cần)
% \logo{\includegraphics[height=1cm]{logo.png}}

\begin{document}

% ====================
% SLIDE 1: TITLE
% ====================
\frame{\titlepage}

% ====================
% SLIDE 2: MỤC LỤC
% ====================
\begin{frame}{Nội dung}
    \tableofcontents
\end{frame}

% ====================
% PHẦN 1: GIỚI THIỆU
% ====================
\section{Giới Thiệu}

\begin{frame}{Giới Thiệu}
    \begin{itemize}
        \item Điểm chính thứ nhất
        \item Điểm chính thứ hai
        \item Điểm chính thứ ba
    \end{itemize}
    
    \vspace{1em}
    
    \textbf{Lưu ý:} Bạn có thể thay đổi nội dung này theo ý muốn.
\end{frame}

\begin{frame}{Bối Cảnh Vấn Đề}
    \begin{block}{Vấn đề}
        Mô tả vấn đề cần giải quyết tại đây.
    \end{block}
    
    \begin{alertblock}{Thách thức}
        Các thách thức chính:
        \begin{enumerate}
            \item Thách thức 1
            \item Thách thức 2
            \item Thách thức 3
        \end{enumerate}
    \end{alertblock}
\end{frame}

% ====================
% PHẦN 2: PHƯƠNG PHÁP
% ====================
\section{Phương Pháp}

\begin{frame}{Phương Pháp Tiếp Cận}
    \begin{columns}
        % Cột bên trái
        \column{0.5\textwidth}
        \textbf{Bước 1:}
        \begin{itemize}
            \item Chi tiết bước 1
            \item Thông tin bổ sung
        \end{itemize}
        
        \textbf{Bước 2:}
        \begin{itemize}
            \item Chi tiết bước 2
            \item Thông tin bổ sung
        \end{itemize}
        
        % Cột bên phải
        \column{0.5\textwidth}
        % Thêm hình ảnh ở đây (uncomment khi đã upload ảnh)
        % \includegraphics[width=\textwidth]{diagram.png}
        
        \begin{exampleblock}{Ví dụ}
            Đây là một ví dụ minh họa.
        \end{exampleblock}
    \end{columns}
\end{frame}

\begin{frame}{Công Thức Toán Học}
    % Ví dụ công thức inline
    Công thức Einstein nổi tiếng: $E = mc^2$
    
    \vspace{1em}
    
    % Ví dụ công thức display
    Phương trình bậc hai:
    \begin{equation}
        x = \frac{-b \pm \sqrt{b^2 - 4ac}}{2a}
    \end{equation}
    
    \vspace{1em}
    
    % Ví dụ align
    \begin{align}
        f(x) &= x^2 + 2x + 1 \\
        &= (x + 1)^2
    \end{align}
\end{frame}

% ====================
% PHẦN 3: KẾT QUẢ
% ====================
\section{Kết Quả}

\begin{frame}{Kết Quả Thực Nghiệm}
    \begin{table}
        \centering
        \begin{tabular}{|c|c|c|}
            \hline
            \textbf{Phương pháp} & \textbf{Độ chính xác} & \textbf{Thời gian} \\
            \hline
            Phương pháp A & 95\% & 10s \\
            Phương pháp B & 92\% & 5s \\
            Phương pháp C & 98\% & 15s \\
            \hline
        \end{tabular}
        \caption{So sánh các phương pháp}
    \end{table}
\end{frame}

\begin{frame}{Biểu Đồ Kết Quả}
    % Uncomment khi đã có ảnh biểu đồ
    % \begin{figure}
    %     \centering
    %     \includegraphics[width=0.8\textwidth]{chart.png}
    %     \caption{Biểu đồ so sánh kết quả}
    % \end{figure}
    
    \begin{center}
        \textit{[Thêm biểu đồ/hình ảnh ở đây]}
    \end{center}
    
    \textbf{Nhận xét:}
    \begin{itemize}
        \item Phương pháp C cho kết quả tốt nhất
        \item Phương pháp B nhanh nhất nhưng độ chính xác thấp hơn
    \end{itemize}
\end{frame}

% ====================
% PHẦN 4: KẾT LUẬN
% ====================
\section{Kết Luận}

\begin{frame}{Tóm Tắt}
    \begin{itemize}
        \item<1-> Đóng góp chính thứ nhất
        \item<2-> Đóng góp chính thứ hai  
        \item<3-> Đóng góp chính thứ ba
    \end{itemize}
    
    \vspace{1em}
    
    \uncover<4->{
        \begin{block}{Kết luận}
            Tóm tắt kết luận chính của bài trình bày.
        \end{block}
    }
\end{frame}

\begin{frame}{Hướng Phát Triển}
    \textbf{Công việc tương lai:}
    \begin{enumerate}
        \item Cải tiến thuật toán
        \item Mở rộng tập dữ liệu
        \item Áp dụng vào thực tế
    \end{enumerate}
    
    \vspace{2em}
    
    \begin{center}
        \Huge{\textbf{Cảm ơn!}}
        
        \vspace{1em}
        
        \normalsize
        \textit{Câu hỏi và thảo luận}
    \end{center}
\end{frame}

% ====================
% SLIDE BỔ SUNG: TÀI LIỆU THAM KHẢO
% ====================
\begin{frame}[allowframebreaks]{Tài Liệu Tham Khảo}
    \begin{thebibliography}{99}
        \bibitem{ref1} Tác giả A. (2024). \textit{Tiêu đề bài báo}. Tên hội nghị/tạp chí.
        
        \bibitem{ref2} Tác giả B. (2023). \textit{Tiêu đề sách}. Nhà xuất bản.
        
        \bibitem{ref3} Tác giả C, et al. (2025). \textit{Bài nghiên cứu}. Journal Name, vol. 10, pp. 123-145.
    \end{thebibliography}
\end{frame}

\end{document}
