\documentclass[12pt,a4paper]{article}
\usepackage{fontspec}
\usepackage{polyglossia}
\setdefaultlanguage{vietnamese}
\usepackage{amsmath}
\usepackage{xcolor}
\usepackage{pgffor}

% Định nghĩa lệnh tính Fibonacci đệ quy
\newcommand{\fib}[1]{%
  \ifnum#1=0
    0%
  \else
    \ifnum#1=1
      1%
    \else
      \numexpr\fib{\numexpr#1-1\relax}+\fib{\numexpr#1-2\relax}\relax
    \fi
  \fi
}

% Định nghĩa lệnh tính Fibonacci lặp (hiệu quả hơn)
\newcounter{fiba}
\newcounter{fibb}
\newcounter{fibc}
\newcounter{fibi}

\newcommand{\fibonacci}[1]{%
  \ifnum#1=0
    0%
  \else
    \ifnum#1=1
      1%
    \else
      \setcounter{fiba}{0}%
      \setcounter{fibb}{1}%
      \setcounter{fibi}{2}%
      \loop
        \setcounter{fibc}{\thefiba}%
        \addtocounter{fibc}{\thefibb}%
        \setcounter{fiba}{\thefibb}%
        \setcounter{fibb}{\thefibc}%
        \addtocounter{fibi}{1}%
        \ifnum\thefibi<#1
      \repeat
      \addtocounter{fibi}{-1}%
      \ifnum\thefibi=#1
        \thefibc
      \fi
    \fi
  \fi
}

\title{Lập trình trong LaTeX: Dãy Fibonacci}
\author{Ví dụ minh họa}
\date{\today}

\begin{document}

\maketitle

\section{Giới thiệu}

LaTeX không chỉ là công cụ sắp chữ mà còn có khả năng lập trình mạnh mẽ. Dưới đây là các ví dụ về cách tính và hiển thị dãy số Fibonacci.

\section{Dãy Fibonacci}

Dãy Fibonacci được định nghĩa như sau:
\begin{equation}
F_n = \begin{cases}
0 & \text{nếu } n = 0\\
1 & \text{nếu } n = 1\\
F_{n-1} + F_{n-2} & \text{nếu } n \geq 2
\end{cases}
\end{equation}

\section{Phương pháp 1: Sử dụng vòng lặp \texttt{\textbackslash foreach}}

\subsection{In 15 số Fibonacci đầu tiên}

\begin{center}
\begin{tabular}{|c|c|}
\hline
\textbf{n} & \textbf{$F_n$} \\
\hline
\foreach \n in {0,1,...,14}{%
  \n & \fibonacci{\n} \\
  \hline
}
\end{tabular}
\end{center}

\section{Phương pháp 2: In dãy Fibonacci dạng liệt kê}

Các số Fibonacci từ $F_0$ đến $F_{20}$:

\[
\foreach \n in {0,1,...,20}{%
  F_{\n} = \fibonacci{\n}
  \ifnum\n<20, \quad \fi
}
\]

\section{Phương pháp 3: Hiển thị dạng danh sách}

\begin{itemize}
\foreach \n in {0,1,...,12}{%
  \item Số Fibonacci thứ \n{} là: \textcolor{blue}{\textbf{\fibonacci{\n}}}
}
\end{itemize}

\section{Phương pháp 4: Tạo đồ thị tỷ lệ}

Ta có thể so sánh các số Fibonacci:

\begin{center}
\foreach \n in {1,2,...,10}{%
  $F_{\n}$: \foreach \i in {1,...,\fibonacci{\n}}{\textcolor{red}{$\bullet$}}
  \par\smallskip
}
\end{center}

\section{Công thức Binet}

Số Fibonacci thứ $n$ cũng có thể tính bằng công thức:
\begin{equation}
F_n = \frac{\phi^n - \psi^n}{\sqrt{5}}
\end{equation}
trong đó $\phi = \frac{1+\sqrt{5}}{2}$ (tỷ lệ vàng) và $\psi = \frac{1-\sqrt{5}}{2}$.

\section{Ví dụ tính toán cụ thể}

\begin{align*}
F_{10} &= \fibonacci{10} \\
F_{15} &= \fibonacci{15} \\
F_{20} &= \fibonacci{20} \\
F_{25} &= \fibonacci{25}
\end{align*}

\section{Kết luận}

LaTeX cung cấp nhiều cách để lập trình:
\begin{itemize}
\item Sử dụng các lệnh điều kiện (\texttt{\textbackslash ifnum})
\item Sử dụng vòng lặp (\texttt{\textbackslash loop}, \texttt{\textbackslash foreach})
\item Sử dụng counter để lưu trữ biến
\item Sử dụng \texttt{\textbackslash numexpr} để tính toán số học
\end{itemize}

\end{document}